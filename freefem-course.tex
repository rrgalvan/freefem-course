% Created 2011-07-10 dom 13:39
\documentclass[11pt]{article}
\usepackage[latin1]{inputenc}
%\usepackage[T1]{fontenc}
\usepackage{fixltx2e}
\usepackage{graphicx}
\usepackage{longtable}
\usepackage{float}
\usepackage{wrapfig}
\usepackage{soul}
\usepackage{textcomp}
\usepackage{marvosym}
\usepackage{xspace}
\usepackage{latexsym}
\usepackage{amssymb}
\usepackage[usenames,dvipsnames,svgnames,table]{xcolor}
\usepackage{hyperref}
\definecolor{darkblue}{rgb}{0.0,0.1,0.3}
\definecolor{darkgreen}{rgb}{0.0,0.35,0.15}
\definecolor{darkred}{rgb}{0.3,0.1,0.0}
\def\refColor{darkgreen}
\hypersetup{%
  colorlinks,
  linkcolor=\refColor,
  urlcolor=\refColor,
  anchorcolor=\refColor,
  citecolor=\refColor
}
\usepackage[margin=2cm]{geometry}
\tolerance=1000
\usepackage{listings}
\usepackage[spanish]{babel}
\usepackage{matematicas,ff++listings}
\usepackage{pgf}

\providecommand{\alert}[1]{\emph{#1}}
\providecommand{\structure}[1]{#1}

\newcommand{\FF}{\textit{FreeFem++}\xspace}
\newcommand{\FFcs}{\textit{FreeFem-cs}\xspace}
\renewcommand{\P}{\mathcal{P}_}
\newcommand{\R}{{\mathbb R}}

\title{Introducci�n pr�ctica al M�todo de los Elementos Finitos con \FF}
\author{J. Rafael Rodr�guez Galv�n}
\date{\today}


\begin{document}

\maketitle

\begin{center}
  \includegraphics[width=0.07\textwidth]{./logouca.png}\par
  \emph{M�ster en Matem�ticas} \par \emph{Universidad de C�diz}.
  \par\vfill
  \includegraphics[width=0.1\textwidth]{./cc-by-sa.png}
  \par\medskip
  \begin{small} \em
    Este documento tiene licencia libre. En concreto, se permite su
    uso bajo las condiciones de la licencia
    \\
    \href{http://creativecommons.org/licenses/by-sa/3.0/}{Creative
      Commons Atribuci�n-Compartir Igual}.
  \end{small}
\end{center}

\setcounter{tocdepth}{2}
\tableofcontents
\vspace*{1cm}

\include{parte1}
\include{parte2}

\end{document}
