% Created 2011-07-10 dom 13:39
\documentclass[12pt]{article}
\usepackage[latin1]{inputenc}
%\usepackage[T1]{fontenc}
\usepackage{fixltx2e}
\usepackage{graphicx}
\usepackage{longtable}
\usepackage{float}
\usepackage{wrapfig}
\usepackage{soul}
\usepackage{textcomp}
\usepackage{marvosym}
\usepackage{xspace}
\usepackage{latexsym}
\usepackage{amssymb}
\usepackage[usenames,dvipsnames,svgnames,table]{xcolor}
\usepackage{hyperref}
\definecolor{darkblue}{rgb}{0.0,0.1,0.3}
\definecolor{darkgreen}{rgb}{0.0,0.35,0.15}
\definecolor{darkred}{rgb}{0.3,0.1,0.0}
\def\refColor{darkgreen}
\hypersetup{%
  colorlinks,
  linkcolor=\refColor,
  urlcolor=\refColor,
  anchorcolor=\refColor,
  citecolor=\refColor
}
\usepackage[hmargin={4cm,3cm}]{geometry}
\tolerance=1000
\usepackage{listings}
\usepackage{math,ff++listings}
\usepackage{pgf}

\providecommand{\alert}[1]{\emph{#1}}
\providecommand{\structure}[1]{#1}

\newcommand{\FF}{\textit{FreeFem++}\xspace}
\newcommand{\FFcs}{\textit{FreeFem-cs}\xspace}
\renewcommand{\P}{\mathcal{P}_}
\newcommand{\R}{{\mathbb R}}

\newcounter{exercise}
\newenvironment{exercise}{%
  \stepcounter{exercise}
  \subsubsection*{Exercise~\theexercise.}}
{}

\title{Introduction to Finite Elements with \FF}
\author{J. Rafael Rodr�guez Galv�n \\ Univesidad de C�diz}
\date{
  % \includegraphics[width=0.13\textwidth]{./logo-unica.png}\par
  ~
  \\[5em]
  \emph{Universit� degli studi di Cagliari}
  \\[0.5em]
  \emph{\today}
  \\[9em]
  \includegraphics[width=0.12\textwidth]{./cc-by-sa.png}
  \\[1em]
  \begin{small} \em
    \href{http://creativecommons.org/licenses/by-sa/3.0/}{Creative
      Commons Attribution-Share\_Alike License, \\ 
      \scriptsize\url{http://creativecommons.org/licenses/by-sa/3.0/}}.
  \end{small}
}

% -----------------------------------
\begin{document}
% -----------------------------------

\maketitle

%\large %%%%%%%%%%%%%%%%%%%%%%%%%%%%%%%%

\newpage
\setcounter{tocdepth}{2}
\tableofcontents
\vspace*{1cm}

\section{Introduction}
\label{sec:introduction}

\subsection{The \FF environment}
\label{sec:ff-environment}

\alert{\FF},\footnote{\url{http://www.freefem.org/ff++}} is a package
for \emph{numerical approximation of the solution of PDE} (partial
differential equations), both 2D and 3D, by means of the \textit{Finite
Element Method} (FEM) \emph{elementos finitos}. \FF is composed of:

\begin{itemize}

\item An interpreted \alert{programming Language}:
  \begin{itemize}
  \item Oriented to fast specification of those problems which can be
    described by means of (linear and steady) PDE, and  also  to resolution
    of those problems, using FEM.
  \item Allows easy implementation of complex problems (nonlinear,
    transient,...)
  \end{itemize}

\item An \alert{interpreter} for that language.
  \begin{itemize}

  \item Programs in FreeFem++ are interpreted (not compiled) at
    runtime. In this sense, FreeFem++ is a \textit{scripting} language
    (like Python, Matlab/Octave, Perl, and others).
  \item \FF is \alert{open source}/\alert{free software} (GNU/GPL licenese).
  \item There are different versions: \texttt{FreFem++},
    \texttt{FreeFem++-nw}, \texttt{FreeFem++-mpi},\ldots{}
  \end{itemize} % ends low level
\end{itemize} % ends low level

\FF is mainly developed by \textbf{F. Hetch}.
\begin{center}
  \includegraphics[width=0.3\linewidth]{./Hecht-portrait}
\end{center}

\subsection{What can you do with \FF?}
Some examples related to Stokes equations:
\begin{equation*}
  \left\{
  \begin{aligned}
    -\nu\Delta \mathbf{u}+ \nabla p &= f 
    \\ 
    \nabla\cdot \mathbf{u} & = 0
    \\ & + \text{boundary conditions},
  \end{aligned}
  \right.
\end{equation*}
where the unknowns are $\mathbf{u}=(u,v):\Omega\to\R$ (velocity field of fluid)
and $p:\Omega\to\R$ pressure in each point of the domain.
\begin{itemize}
\item \href{videos/mediterraneo2.avi}{{}The Stokes/Navier-Stokes equations in Mediterranean sea}
\item Why I like numerical simulation (as mathematician): it helps you
  to understand
  theory, \href{videos/eps-to-zero-P2-P1-streamlines-20s.avi}{in
    this case, instability of a numerical scheme}
\end{itemize}


\subsection{Characteristics  of the \FF Language}
\begin{itemize}

\item Inspired by \alert{C/C++}.
  \begin{itemize}
  \item \alert{Similarities}: Syntax, strong typing\ldots{}
  \item \alert{Does not include}: Pointers, object orienting, \ldots{}
  \end{itemize}

\item \alert{Oriented} to numerical simulation using the finite element method.
  Possibilities:
  \begin{itemize}
  \item Definition of the geometry of a problem and 2D/3D meshing
    (although \FF is not a CAD/CAE environment and then, if one wish
    to use complex geometries, it is necessary to employ external tools.
  \item Variety of available \alert{finite elements}:
    $P_k$--Lagrange, $P_1$--bubble, $P_1$ discontinuous,
    Raviart-Thomas\ldots{}
  \item Flexibility for definition of problems which can be formulated
    in terms of PDE (and expressed by a \alert{variational formulation})
  \item Automation of the task of assemble FEM matrices involved in
    underlying linear systems, so that this task is transparent to the
    user.
  \item Several algorithms for \alert{resolution of those linear
      systems}: LU, Cholesky, Crout, CG, GMRES, UMFPACK\ldots{}
  \item Facilities for \alert{post-processing and 2D/3D
      visualizaci�n}.  Although \FF no is not specialized in
    scientific visualizaci�n, it can be complemented with external
    tools for high-quality graphics.
  \item Many other issues:
    \begin{itemize}
    \item Excellent documentation (with a plenty of examples and
      tutorials):
      \url{http://www.freefem.org/ff++/ftp/freefem++doc.pdf}.
    \item \textit{Matlab}--like matrix manipulation (or
      Matlab/Octave/Python/Fortran--like). 
    \item Automatic interpolation between meshes, adaptive refinement,...
    \item Parallel (with MPI) version available
      (\texttt{FreeFem++-mpi}) in UNIX systems.
    \end{itemize}
  \end{itemize}

\end{itemize} % ends low level

\section{Installation and first steps}
\label{sec:inst-first-steps}

The \FF package includes an interpreter for execution of code (code
which is written in \FF language) and also some additional tools. But
the standard edition does not include an integrated environment (with
editor, error feedback, syntax highlighting, etc.).  The user is
allowed to choose its preferred editor between different
possibilities, as we comment below.

In each operative system, there are different possibilities for the
selection of an adequate editor (for instance \textit{Crimson Editor}
or \textit{Notepad++} in Windows or \texttt{Emacs} in GNU/Linux, MacOS
or Windows). The installation process is described, for some of them,
in the \FF manual.

Anyway, for a first approach to \FF, we recommend a tool called \FFcs,
\url{http://www.ann.jussieu.fr/~lehyaric/ffcs/}, \FF and an integrated
environment providing adequate characteristics and an easy
installation process. Of course, advanced users may prefer other
options.

\subsection{\FFcs: an integrated environment for \FF}
\label{sec:freefem-CS}

\FFcs (\textit{CS} $\leftarrow$ Client/Server) is package which contains
both \FF and an integrated environment for \FF providing an intuitive
interface. It adds to \FF the following goodies:
\begin{itemize}
\item Integrated interface, aimed at making users comfortable.
\item Color-coded editor.
\item Automatic highlighting of \FF.
\item Compilation errors, linked back to the EDP source code.
\item Integrated graphics area for 2d and 3d.
\item Online help including documentation in HTML.
\item Multi-platform (Windows--GNU/Linux--MacOS).
\end{itemize}


\subsubsection{Installation}
For installation, you can get your preferred version from
the ``\textit{Download}'' link
(\url{http://www.ann.jussieu.fr/~lehyaric/ffcs/install.php}) and
follow the specific instructions for each  platform (which consist in
only a few steps). For instance:

\begin{itemize}
\item \textit{Windows}: Execute the installation program and follow
  usual steps. Once installed, click on the \textit{FreeFem++-cs} icon
  and start using the application.
\item \textit{GNU/Linux} (\textit{Ubuntu}, \textit{Debian} and
  others): Decompress the \texttt{.tgz} in your preferred location (for
  instance, in the desktop). Run the program \textit{FreeFem++-cs}
  (located in the folder created when decompress). 
\item \textit{MacOS}: Decompress the \texttt{.zip} file in your
  preferred location (e.g. in the desktop). Run \textit{FreeFem++cs}.
\end{itemize}

\begin{exercise}
  Download \FFcs from the web (choose the adequate version for your
  preferred operative system) and install it.
\end{exercise}

\begin{exercise}
  Open the FreeFem++ manual,
  \url{http://www.freefem.org/ff++/ftp/freefem++doc.pdf}, and search
  for recommended editors (in Section 1.1). Choose an editor, install
  and configure it for use with \FF.
\end{exercise}

\subsubsection{First steps with \FFcs}

\FFcs is composed of three different panels:
\begin{enumerate}
\item Editor with syntax highlighting (left).
\item Messages returned by the interpreter (bottom).
\item Graphics generated by our numerical simulation (right).
end{enumerate}
\end{enumerate}

Other Characteristics:

\begin{itemize}
\item The FreeFem++ script (program) can be run at any time by
  clicking in the \texttt{Run} buttom (top left), or pressing
  \texttt{Ctrl+Shift+R}.
\item The script can be stopped at any time by clicking the
  \texttt{Kill} button (top left).
\item Dragging a \FF script file into \FFcs (icon or editor) makes \FFcs
  edit that script.
\end{itemize}

In the following exercise, we write a very simple FreeFem++ script
which (a) plots a simple mesh in the unit square
$[0,1]\times[0,1]$, which is defined by two subintervals (of $[0,1]$)
in the $x$ axis and also two subintervals in $y$.
\begin{exercise}
  Write the following code and run it. Test that a graphic appears in the
  right panel and a message is written in the bottom panel.
  \lstset{language=freefem++}
\begin{lstlisting}
mesh Th = square(2,2); // Declare a mesh object and build it
plot(Th);
cout << "Hello world!" << endl;
\end{lstlisting}
\end{exercise}

Former code may result quite familiar to C++ programmers.

\subsection{A first realistic example}

Now we are going to solve for the first time a PDE system by means of
the FEM method. In further sections more complex problems are studied.
Specifically, we solve the following example (Poisson equation with
Dirichlet boundary conditions):
\begin{equation*}
  \begin{cases}
    \text{Find } u:\bar\Omega \rightarrow \Rset 
    \text{ such  that}
    \\\noalign{\medskip}
    \begin{aligned}
      -\Delta u &= f \quad\text{in } \Omega,
      \\
      u &= 0 \quad\text{on } \partial\Omega.
    \end{aligned}
  \end{cases}
\end{equation*}

For that purpose, we proceed as follows:
\begin{description}
\item[Step 1.] Express the problem in (discrete) variational formulation:
  \vspace{-2ex}
  \begin{equation*}
    \begin{cases}
      \text{Find } u:\bar\Omega \rightarrow \Rset 
      \text{ such that }
      \\\noalign{\medskip}
      \begin{aligned}
        -\Delta u &= f \quad\text{in } \Omega,
        \\
        u &= 0 \quad\text{on } \partial\Omega.
      \end{aligned}
    \end{cases}
    \qquad \leadsto \qquad
    \begin{cases}
      \text{Find $u_h\in X_{h}$ such that} 
      \\\noalign{\medskip}
      \begin{aligned}
        \int_\Omega \nabla u_h \cdot \nabla v_h = \int_\Omega f \cdot
        v_h,
        \quad\text{for each } v_h\in X_{h}.
      \end{aligned}
    \end{cases}
  \end{equation*}

\item[Step 2.] Translate the variational formulation into \FF
  language. 
  Supposing that the domain, $\Omega$, is given by the unit circle, we
  can write the following script :
  \lstset{language=freefem++}
\begin{lstlisting}
border gamma(t=0, 2*pi) { x=cos(t); y=sin(t); } // Define boundary
mesh Th = buildmesh(gamma(20)); // Build mesh with 20 intervals on boundary
fespace Xh(Th,P1); // Define P1 finite element space
Xh u,v; // Build two functions in this space (unknown and test function)
func f = exp(x)*sin(y);  // Right hand side
solve Dirichlet(u,v) =   // Define variational formulation  
  int2d(Th) ( dx(u)*dx(v) + dy(u)*dy(v) )
  - int2d(Th) ( f*v )
  + on( gamma, u=0 );
plot(u);  // Show results
\end{lstlisting}

\end{description}

This piece of code contains the fundamentals of FEM with \FF.
\begin{enumerate}
\item In lines 1 and 2 we define the circular domain. The technique
  consists of the parametrization of the boundary. Any domain with
  parametrizable boundary can be easily introduced in \FF.  For other
  domains, one has to use a specific tool for mesh construction and
\item In line 3 we define the FE (finite element) space,
  $\P1$--Lagrange in this case, and in line 4 we declare two variables
  in this space. We intend to use the first one, \texttt{u}, as the FE
  unknown (the \texttt{trial} function), and the second one, \texttt{v} 
  as the \texttt{test} function.
\item In line 5 we define a function. Note that standard variables
  \texttt{x} and \texttt{y} are predefined and must not be declared.
\item In lines 6--9 we solve the variational problem. Note that those
  lines constitute a quasi-literal transcription of the variational
  problem formulated in \texttt{Step 1}. Some comments:
  \begin{enumerate}
  \item By default, PDE operators like gradient ($\nabla$) are not
    predefined (although they can be defined using macros, as we see
    in a further section). So one must use the operators \texttt{dx}
    ($\frac{\partial}{\partial x}$) and \texttt{dy}
    ($\frac{\partial}{\partial y}$). For 3D programs, also \texttt{dz}
    can be employed.
  \item Dirichlet conditions are imposed as the ``artificial'' sum of
    a term to the bilinear form.
  \end{enumerate}

\item Finally, in line 10 we plot the obtained solution. Scalar data
  (as, in this case, $u$) is plotted by contour plots, while vector
  data is plotted as arrow field (for instance, the velocity unknown
  in the context of Stokes equations).
\end{enumerate}


\subsection{Saving to VTK for high-quality graphics}

VTK consists of an open source C++ library for visualization of
different types of data (scalar, vector, tensor, etc.). Last versions
of FreeFem++ include a module (called \texttt{iovtk}) which can be
loaded for use VTK. This way users can save any FE function to a
\texttt{.vtk} file and then employ any of the available advanced
applications for manipulation and visualization of the data contained
in that file. In section~\ref{sec:brief-intro-paraview} we delve into
one of those applications, called
Paraview\footnote{\url{http://www.paraview.org/}}.

The following code can be appended to the script above for saving the
solution, $u$, into a VTK file. 
\lstset{language=freefem++}
\begin{lstlisting}
load "iovtk";
savevtk("/tmp/output.vtk", Th, u, dataname="Temperature");
\end{lstlisting}

The module \texttt{iovtk} provides the function
\texttt{savevtk}. Their compulsory parameters are: (1) name of the
output VTK file, (2) name of the mesh, (3) FE function to be
saved. More than one function can be saved in the same file, as we
will see below. The last parameter is optional (but recommended) and
provides a name for each saved function. In this case, assuming that
the solution represents the equilibrium state of a heating experiment,
the only data set is called ``Temperature''. We can use this name to
access the data in the future (for instance using Paraview).


\section{Addressing more complex problems}
\label{sec:complex-problems}

In the section we go beyond the Poisson problem and generalize it in
different ways.
\begin{enumerate}
\item Introducing other kind of boundary conditions (Neumann b.c.)
\item Handling transient (time dependent) problems (Heat equation).
\end{enumerate}

\subsection{Poisson problem with mixed boundary conditions}
\label{sec:poisson-problem-with-mixed-bc}

We set the problem: given
\begin{itemize}
\item $\Omega\subset\R^2$, with smooth piecewise boundary, where we
  distinguish two zones: 
  $\partial\Omega=\Gamma_0\cup\Gamma_1$
\item $\nu>0$, \quad $f:\Omega\to\R$, \quad $g:\Gamma_0\to\R$
\end{itemize}
Given $f: \Omega \to \Rset$, $g_0:\Gamma_0\to\Rset$ and
$g_2:\Gamma_1\to\Rset$, we try to find $u:\bar\Omega \rightarrow \R$
such that:
\begin{equation}
  \label{eq:poisson-mixto}
  \begin{cases}
    \begin{aligned}
      -\nu\Delta u &= f \quad \text{ in } \Omega, \\
      u &= g_0 \quad \text{ on } \Gamma_0, \\
      \frac{\partial u}{\partial n} &= g_1 \quad \text{ on } \Gamma_1.
    \end{aligned}
  \end{cases}
\end{equation}
Then one has have a (non homogeneous) Dirichlet b.c. in $\Gamma_0$ and
a Neumann b.c on $\Gamma_1$. Remember that last condition means, means
$\nabla u \cdot n=g_1$, where $n$ is the exterior normal vector.

\begin{itemize}
\item The theory for \textbf{non-homogeneous Dirichlet} conditions,
  $u|_{\Gamma_0}=g_0$, is based on writing the solution as
  $$
  u = u_0 + u_D,
  $$
  where $u_0$ is a solution of the homogeneous problem
  ($u|_{\Gamma_0}=0$) while $u_D$ verifies $u|_{\Gamma_0}=g_0$.
  In practice, for Dirichlet conditions one proceed as follows:
  \begin{enumerate}
  \item Build the FE linear system $Ax=b$, where $A$ comes from a
    bilinear form, $a(\cdot,\cdot)$ and $b$ comes from a linear form,
    $L(\cdot)$. Both of $A$ and $b$ are, typically constructed by
    quadrature formulae in triangles.
  \item Select the rows of $A$ and $b$ which correspond to equations
    relative to degrees of freedom (for instance vertices of the
    triangles) placed on $\Gamma_D$. Then modify them, imposing
    explicitly the value of $u$ on that degrees of freedom.
  \end{enumerate}
  This issue is automatized by \FF and then we are not going
  deeper.

  % writing the variational formulation, we  assume the homogeneous
  % Dirichlet condition
  % $$u=0 \text{ on } \Gamma_1,$$
  % and imposing later $u=g_0$ on $\Gamma_0$. 
\item But Neumann boundary conditions appear in a natural way in the
  variational formulation. Specifically, when the Green formula
  (integration by parts) is applied, one gets the following problem:
  \begin{equation*}
      \begin{cases}
      \text{Find $u_h\in U_{h}$ such that} 
      \\\noalign{\medskip}
      \begin{aligned}
        \int_\Omega \nabla u_h \cdot \nabla v_h = \int_\Omega f\, v_h 
        - \int_{\Gamma_1} g_1\, v_h
        \quad\text{for each } v_h\in U_{h},
      \end{aligned}
    \end{cases}
  \end{equation*}
  being $U_h$ the set of functions $u_h:\Omega\to\Rset$ such that
  \begin{itemize}
  \item $u_h|_T \in \mathbb{P}_k[x]$ (polynomials of degree $k$) for
    all $T\in \mathcal{T}_h$ and
  \item $u_h|_{\Gamma_0}=0$.
  \end{itemize}
\end{itemize}

\subsubsection{FreeFem++ programming}

\paragraph{Step 1: Pre-processing}

\lstset{language=freefem++}
\begin{lstlisting}
  // 1. Pre-proceso

  // 1.1. Mesh
  border gamma0(t=2*pi, 0) { x=1.5*cos(t); y=sin(t); }
  border gamma1(t=0, 2*pi) { x=4*cos(t); y=4*sin(t); }
  mesh Th = buildmesh(gamma1(40)+gamma0(30));
  plot(Th, wait=1);

  // 1.2. FE space and functions 
  fespace Vh(Th,P1);
  Vh u,v;

  // 1.3. Definition of data
  real nu=0.3;
  func f=8*(x^2+y^2);
  func g=400;
\end{lstlisting}

\paragraph{Step 2: Processing}

\lstset{language=freefem++}
\begin{lstlisting}[firstnumber=19]
  solve example1(u,v)=
  // Bilinear form:
  int2d(Th)( nu*( dx(u)*dx(v) + dy(u)*dy(v) ))
  // Linear lorm:
  - int2d(Th)( f*v ) 
  - int1d(Th, gamma1)( nu*g*v )
  // Dirichlet boundary condition
  + on(gamma0, u=g);
\end{lstlisting}

\paragraph{Step 3: Post-processing}

\lstset{language=freefem++}
\begin{lstlisting}
  // 3. Post-proceso
  plot(u, value=1, fill=1, wait=1);
\end{lstlisting}

\subsection{The heat equation}
\label{sec:heat-equation}


Former examples were steady, namely time independent. Now we set the
first transient problem, where the time variable is present. given
\begin{itemize}
\item $\Omega\subset\R^2$, with smooth piecewise boundary,  $\partial\Omega=\Gamma_0\cup\Gamma_1$
\item $T>0$: final time, $n$: number of time iterations in $[0,T]$.
\item $u_0$: $\Omega\to\R$: temperature at initial time.
\item $\nu>0$, $f:\Omega\times(0,T)\to\R$ (heat source in the domain).
  $u_{\text{ext}}:\Gamma_1\times(0,T)\to\R$ (heat source on boundary $\Gamma_1$).
\end{itemize}

For time discretization, consider $n+1$ time instants in $[0,T]$, $t_k=
\verb|dt|\cdot k$, $k=0, ..., n$, being $\verb|dt|=T/n$ the time
step.

The Euler method reads:
\begin{itemize}
\item Initialization: for $k=0$, take $u^0=u(t=0)=u_0$
\item Step $k$: given $u(t_k)$, find $u^{k+1}\in U_h$ (defined in
  section~\ref{sec:poisson-problem-with-mixed-bc}) such that 
% Find \quad $u:\bar\Omega\times(0,T) \rightarrow \R$, representing
% temperature evolution in $\Omega$, such that:
\begin{equation*}
  a(u_h,v_h) = b(v_h) \quad \forall v_h\in U_h,
\end{equation*}
where
\begin{equation*}
    \begin{aligned}
      a(u,v)&=\int_\Omega\frac{u^{k+1}}{dt} v + \nu\int_\Omega \nabla
      u \cdot \nabla v,
      \\
      b(v)  &= \int_\Omega f \cdot v
        +\int_\Omega \frac{u^{k}}{dt} v
    \end{aligned}
\end{equation*}
% \begin{equation}
%   \begin{cases}
%     \frac{\partial u}{\partial t} -\nu\Delta u = f \text{ in } \Omega\times(0,T), \\
%     u = g_0 \text{ on } \Gamma_0\times(0,T), \\
%     u = g_1 \text{ on } \Gamma_1\times(0,T), \\    
%   \end{cases}
% \end{equation}

  
 \end{itemize}
% ecuaci�n anterior, obtendremos el siguiente sistema en tiempo: En la
% etapa inicial ($k=0$), tomamos $u^0=u(t=0)=u_0$. Conocidos los valores
% $u^0,...,u^k$, que aproximan a los valores de $u(t_0),...,u(t_k)$, en
% la etapa $k+1$, calculamos $u^{k+1}$ como soluci�n de: hallar $u\in
% H^1(\Omega)$ tal que $u=g_0$ en $\Gamma_0$, $u=g_1$ en $\Gamma_1$ y
% \begin{equation*}
%   \left\{
%     \begin{aligned}
%       &\int_\Omega\frac{u^{k+1}-u^k}{dt} v + \nu\int_\Omega \nabla u \cdot \nabla v 
%         = \int_\Omega f \cdot v
%       \\
%       &\text{para todo $v\in H_0^1(\Omega)$.}
%     \end{aligned}
%   \right.
% \end{equation*}
% Como se coment� antes, el problema anterior puede reformulado para
% adaptarlo al marco de Lax-Milgram y, as�, demostrar que existe una
% �nica soluci�n d�bil. A continuaci�n, si pasamos al segundo miembro
% los datos conocidos, podemos identificar las formas bilineal y lineal
% siguientes: 
% \begin{equation*}
%     \begin{aligned}
%       a(u,v)&=\int_\Omega\frac{u^{k+1}}{dt} v + \nu\int_\Omega \nabla
%       u \cdot \nabla v,
%       \\
%       b(v)  &= \int_\Omega f \cdot v
%         +\int_\Omega \frac{u^{k}}{dt} v
%     \end{aligned}
% \end{equation*}
% y a partir de ah� escribir el programa que se muestra a continuaci�n, que
% calcular� la soluci�n en cada etapa de tiempo (a trav�s de un bucle
% \verb|for|) y, en cada etapa, grabar� en un fichero postscript la
% soluci�n correspondiente.

\subsubsection{\FF program}

\lstset{language=freefem++}
\begin{lstlisting}
load "iovtk"; // We will output vtk 

// 1. Pre-processing

// 1.1. Mesh
real R=1;
border gamma0(t=0, pi/4)    { x=R*cos(t); y=R*sin(t); }
border gamma1(t=pi/4, 2*pi) { x=R*cos(t); y=R*sin(t); }

int n=30;
mesh Th = buildmesh(gamma0(n)+gamma1(9*n));
plot(Th, wait=1);

// 1.2. FE space and functions
fespace Vh(Th,P1);
Vh u, v;
Vh uold;

macro gradient(u) [dx(u), dy(u)] // End Of Macro

// 1.3. Data definition
real nu=1;
real t=0, T=1;  // Time interval [0,T]
int  N=100; // Number of time iterations
real dt=T/N;  // Time step

func f=0; //8*(x^2+y^2);
// func real g1(real x, real y, real t) {
//   return 40*t;
// }
func real g0(real x, real y, real t) {
  return 100*(1-1./(t+1));
}
func real g1(real x, real y, real t) {
  return 0;
}

func u0=0; // Init

uold = u0;

// 2. Processing

// Declare (but not solve) the heat equation variational problem
problem heatEquation(u,v)=
  // Bilineal form:
  int2d(Th)(
    u*v/dt +
    nu*gradient(u)'*gradient(v) // ' means transpose
  )
  // Linear form
  - int2d(Th)( uold*v/dt + f*v ) 
  - int1d(Th, gamma1) ( g1(x,y,t)*v ) // Neumann boundary condtion

  // Dirichlet boundary condtion
  + on(gamma0, u=g0(x,y,t));

// Time iteration loop
for (int k=0; k<N; ++k ) {

  t = t + dt;    // Increase current time
  heatEquation;  // Solve the PDE variational problem
  uold = u;      // Save solution for next time step
  
  // 3. Post-processing (save to VTK for further displaying with Paraview)
  string filename="/tmp/heat_equation-" + k + ".vtk";
  savevtk(filename, Th, u, dataname="Temperature");
}
\end{lstlisting}
% -------------------

\subsection{The Stokes equations}
\label{sec:stokes}

The Stokes equations can be considered as the linear steady version of
Navier-Stokes equations (which describe the behaviour of a newtoninan
fluid as atmosphere, ocean, flux around vehicles, etc.
\begin{equation*}
  \left\{
  \begin{aligned}
    -\nu\Delta \mathbf{u}+ \nabla p &= f 
    \\ 
    \nabla\cdot \mathbf{u} & = 0
    \\ & + \text{boundary conditions},
  \end{aligned}
  \right.
\end{equation*}
where the unknowns are: $\mathbf{u}=(u,v):\Omega\to\R$ (velocity field
of fluid) and $p:\Omega\to\R$ pressure in each point of the domain.
Thus the first equation must be understood in vectorial way,
specifically, in the 2D case:
\begin{align*}
  \Delta u + \partial_x p &= f_1, \\
  \Delta v + \partial_y p &= f_2,
\end{align*}
where  $f=(f_1,f_2)$.

In this section we show a usual test for the Stokes 2D simultion,
which is know as \textbf{cavity test}. This test is usually run in a
rectangular domain but, in this case, with the purpose of illustrate
the construction in \FF of complex parametric geometries, we have
introduced some holes in the rectangular domain. They are defined by
parametric figures which are known as
conchoids\footnote{\url{http://en.wikipedia.org/wiki/Conchoid_\%28mathematics\%29}}.  

Homogeneous Dirichlet b.c., $(u,v)=(0,0)$, are imposed for
$\mathbf{u}$ on the whole boundary excepting the top line, where we
fix $(u,v)=(1,0)$  (positive horizontal velocity). We use the stable
FE combination  $\P2/\P1$ (polynomials with degree $2$ for velocity
and degree $1$ for pressure).

\subsubsection{Programaci�n con \FF}

\lstset{language=freefem++}
\begin{lstlisting}
// 2D Stokes equations
// Cavity test in a domain with some parametric holes

//,------------------------------------------------
//| STEP 1. Defining the domain and meshing it
//`------------------------------------------------

// Macro for the 2D boundary defining a hole. They are parametric
// curves called "conchoids". In the macro: 
// n = number of 'petals', P = center of the hole
int NMAX=20;
macro conchoid(name, n, P, thelabel) 
 name(i=0,NMAX) {
    real a=1.0, b=2.0;
    real theta = i*2*pi/NMAX;
    real rho = a * cos(n*theta)+b;
    x = P[0] + rho*cos(theta);
    y = P[1] + rho*sin(theta);
    label = thelabel;
} // EOM

// Definition of some conchoids
border conchoid(c2,2,[0,0]   ,0);
border conchoid(c3,3,[-10,0] ,0);
border conchoid(c4,4,[0,0]   ,0);
border conchoid(c5,5,[10,0]  ,0);
border conchoid(c6,6,[0,0]   ,0);
border conchoid(c7,7,[0,0]   ,0);

// External rectangle
real xcoor = 15, ycoor = 5;
border lx1(k=-xcoor,xcoor) { x=k; y=-ycoor; label=1; }
border lx2(k=-xcoor,xcoor) { x=k; y=+ycoor; label=3; }
border ly1(k=-ycoor,ycoor) { x=-xcoor; y=k; label=2; }
border ly2(k=-ycoor,ycoor) { x=+xcoor; y=k; label=2; }

int nx=40, ny=20, nc=50;
mesh Th = buildmesh( ly1(-ny)+lx1(nx)+ly2(ny)+lx2(-nx) 
    + c3(-nc) + c4(-nc) + c5(-nc) );

//,--------------------------------------------------------
//| STEP 2. Resolution of Stokes problem in previous domain
//`--------------------------------------------------------

fespace Uh(Th,P2); Uh u,v,uu,vv;  // Velocity functions
fespace Ph(Th,P1); Ph p,pp;       // Pressure functions

real upperVelocity=1;

macro grad(u) [dx(u), dy(u)] // end of macro
  
// Definition of Stokes problem

problem stokes2d( [u,v,p], [uu,vv,pp], solver=LU) =
    int2d(Th)(
	grad(u)'*grad(uu) + grad(v)'*grad(vv) 
	+ grad(p)'*[uu,vv] + pp*(dx(u)+dy(v)) //'
	- 1e-10*p*pp )
  + on(0,1,2,u=0,v=0) + on(3,u=upperVelocity,v=0);

stokes2d;  // Resolution of Stokes problem

// Save to VTK (for high quality plotting)  
load "iovtk";
savevtk("/tmp/stokes.vtk", Th, [u,v,0], p);
\end{lstlisting}


\appendix

\section{Paraview}
\label{sec:brief-intro-paraview}

ParaView is an open source multiple-platform application for
interactive, scientific visualization. It was developed to analyze
extremely large datasets using distributed memory computing
resources. It can be run on supercomputers to analyze datasets of
terascale as well as on laptops for smaller data.

For visualization of data, that lives in a mesh where the simulation
was performed, there are basically three steps:
\begin{enumerate}
\item \textit{Reading} data into Paraview (from a VTK file)
\item \textit{Filtering}, that is applying one or more filters in order to
  generate, extract or derive features from data.
\item \textit{Rendering} an image from the data and adjusting the
  viewing parameters for improve the final visualization.
\end{enumerate}

This tree steps are controlled through a panel in the right, called
\texttt{Pipeline browser}. The pipeline concept consists on a chain of
modules, starting from the data stored in a file. Each
of them takes in some data, operates on it and presents the result in
a dataset.
From the Paraview users guide:
\begin{quote}\small
  ``Reading data into ParaView is often as simple as selecting
  \texttt{Open} from the \textit{File} menu, and then clicking the
  glowing \textit{Accept} button on the reader's \textit{Object
    Inspector} tab. ParaView comes with support for a large number of
  file formats, and its modular architecture makes it possible to add
  new file readers.  Once a file is read, ParaView automatically
  renders it in a view. In ParaView, a view is simply a window that
  shows data. There are different types of views, ranging from
  qualitative computer graphics rendering of the data to quantitative
  spreadsheet presentations of the data values as text. ParaView picks
  a suitable view type for your data automatically, but you are free
  to change the view type, modify the rendering parameters of the data
  in the view, and even create new views simultaneously as you see fit
  to better understand what you have read in. Additionally, high-level
  meta information about the data including names, types and ranges of
  arrays, temporal ranges, memory size and geometric extent can be
  found in the \textit{Information} tab.''
\end{quote}
Advanced data processing can be done using the Python Programmable
filter with VTK, NumPy, SciPy and other Python modules.

\bigskip

For further details:
\begin{enumerate}
\item Video showing how to use FreeFem++ and Paraview for
  visualization of 2D and 3D cavity tests for the Stokes Equations
  (partially in
  spanish). \url{https://www.youtube.com/watch?v=wChDeo2A03E}
\item Paraview Wikipedia page (in which this appendix is
  based). \url{http://en.wikipedia.org/wiki/ParaView}.
\item Resources in the web, for instance
  \url{http://vis.lbl.gov/NERSC/Software/paraview/docs/ParaView.pdf}.
\item The paraview users guide (how to unleash the beast!)
  \url{http://denali.princeton.edu/Paraview/ParaViewUsersGuide.v3.14.pdf}
\end{enumerate}
\end{document}

%%% Local Variables: 
%%% mode: latex
%%% TeX-master: t
%%% ispell-local-dictionary: "english"
%%% End: 
