\documentclass[11pt]{article}
\usepackage[utf8]{inputenc}
\usepackage[spanish]{babel}
\usepackage[T1]{fontenc}
\usepackage{fixltx2e}
\usepackage{graphicx}
\usepackage{longtable}
\usepackage{float}
\usepackage{wrapfig}
\usepackage{soul}
\usepackage{textcomp}
\usepackage{marvosym}
\usepackage{xspace}
\usepackage{latexsym}
\usepackage{amssymb}
\usepackage[usenames,dvipsnames,svgnames,table]{xcolor}
\usepackage{hyperref}
\definecolor{darkblue}{rgb}{0.0,0.1,0.3}
\definecolor{darkgreen}{rgb}{0.0,0.35,0.15}
\definecolor{darkred}{rgb}{0.3,0.1,0.0}
\def\refColor{darkgreen}
\hypersetup{%
  colorlinks,
  linkcolor=\refColor,
  urlcolor=\refColor,
  anchorcolor=\refColor,
  citecolor=\refColor
}
\usepackage[hmargin={2.5cm,2cm}, vmargin=2.1cm]{geometry}
\tolerance=1000
\usepackage{listings}
\usepackage{ff++listings}
\usepackage{pgf}

\providecommand{\alert}[1]{\textbf{#1}}
\providecommand{\structure}[1]{#1}

\newcounter{exercise}
\newenvironment{exercise}[1][]{%
  \stepcounter{exercise}
  \subsubsection*{Ejercicio~\theexercise. #1}}
{}

\title{Ejercicios}
% \author[$\dagger$]{J. Rafael Rodríguez Galván}
% \affil[$\dagger$]{Departamento de Matemáticas. Universidad de Cádiz. \texttt{rafael.rodriguez@uca.es}}
\date{ ~
    \\[-4em]
  \textsf{\textsc{EDP y Métodos Numéricos}
    \\
    (\textit{Máster
      Interuniversitario en Matemáticas})}
  \\[1em]
    Versión 0.9
    }

% -----------------------------------
% Code listing style
% -----------------------------------
\definecolor{codegreen}{rgb}{0,0.6,0}
\definecolor{codegray}{rgb}{0.5,0.5,0.5}
\definecolor{codepurple}{rgb}{0.58,0,0.82}
\definecolor{backcolour}{rgb}{0.94,0.94,0.95}

\lstdefinestyle{mystyle}{
    % backgroundcolor=\color{backcolour},
    % commentstyle=\color{codegreen},
    % keywordstyle=\color{magenta},
    % numberstyle=\tiny\color{codegray},
    % stringstyle=\color{codepurple},
    basicstyle=\footnotesize\ttfamily,
    % basicstyle=\linespread{1.2}, % Line spacing
    % breakatwhitespace=false,
    % breaklines=true,
    % numbers=left,
    % numbersep=5pt,
    aboveskip=1.0em,
    belowskip=1.0em,
    language=freefem++
}
\lstset{style=mystyle}
% -----------------------------------

\usepackage{pde}
\newcommand{\V}{V}

% -----------------------------------
\begin{document}
% -----------------------------------

\maketitle

\setcounter{tocdepth}{2}
% \tableofcontents
% \vspace*{1cm}

\begin{exercise}[La ecuación de Fokker-Planck]

Consideramos la siguiente ecuación:
\begin{equation}
  \label{eq:flokker-planck}
  \left\{
    \begin{aligned}
    &u_t = \div(\grad u + u \grad\V), \quad \text{en } \Omega\times (0,T),
    \\[0.2ex]
      &\grad u\cdot n =0, \quad \text{en } \partial\Omega\times(0,T),
      \quad u(\xx,0)=u_0(\xx), \quad \xx\in\Omega,
    \end{aligned}
    \right.
\end{equation}
donde $\Omega\subset\Rset^d$ (en la práctica, $d=2$ o $3$) es un abierto
acotado con frontera regular y $T>0$ es un tiempo final dado.

Esta ecuación interviene en distintos ámbitos, como transporte en
semiconductores, física del plasma o dinámica
estelar~\cite{jungel_entropy_2016}. La variable $u=u(\xx,t)\ge 0$
representa densidad, dentro del ámbito considerado, mientras que el
vector $\V(\xx)\in\Rset^d$ es un potencial para el que, por
simplicidad, asumimos $\grad\V \cdot \nn=0$ sobre
$\partial\Omega\times (0,T)$.

Asumimos eque existe una única solución de~(\ref{eq:flokker-planck}) y
que ésta es suficientemente regular para que las siguientes
expresiones tengan un sentido riguroso.


\begin{quote}
  \textit{El objetivo de esta práctica es responder a los siguientes
    apartados. Algunos de ellos son de tipo teórico y otros son
    prácticos. Para los primeros, puedes generar un documento pdf que
    contenga las respuestas. Para lo ejercicios prácticos, debes
    entregar un archivo comprimido que contenga los ficheros
    \texttt{.edp} junto al informe pdf o las gráficas indicadas en
    cada apartado. Opcionalmente, se podrían entregar uno o más vídeos
    (generados con Paraview) mostrando la evolución en 3D de las
    soluciones numéricas obtenidas.}
\end{quote}

\subsection*{Apartado 1.}
\begin{enumerate}
\item Deducir la siguiente formulación variacional
para~(\ref{eq:flokker-planck}) (que tiene sentido en $D'(0,T;\Rset)$):
hallar $u\in H^1(\Omega)$ tal que
\begin{equation}
\label{eq:formulacion-variacional-FP}
\frac{d}{dt} (u(t),v)_{L^2(\Omega)}+ a(u,v)=0, \quad\forall v \in V.
\end{equation}
donde
$$
a(u,v):=(\grad u, \grad v)_{L^2(\Omega)} + (u \grad V, \grad v)_{L^2(\Omega)}.
$$
\item Realizar un programa FreeFem++, llamado \texttt{apartado1b.edp},
  en el que se aproxime la solución de~(\ref{eq:flokker-planck}) en
  $\Omega\times (0,T) = C_2\times(0,1)$, donde $C_2$ es el interior de
  la circunferencia de radio 2, mediante un esquema de tipo Euler
  implícito para la discretización en tiempo y elementos finitos
  $P_1$--Lagrange para la discretización en espacio. Se utilizarán los
  siguientes parámetros:
  \begin{itemize}
  \item $u_0(\xx)=u_0(x,y) =  e^{-10(x^2+y^2)}$
  \item $\V(\xx)=10$.
  \item Número de vértices en la frontera de $C_2$: $n_x=250$.
  \item Número de iteraciones de tiempo: $n_t=50$ (por tanto, el paso
    de tiempo es $k=T/n_t = 1/50$.
  \end{itemize}
  \begin{flushright}
    \begin{quote}\small
      \it Observación: para los datos anteriores, se tiene
      $\grad\V=0$, luego~(\ref{eq:flokker-planck}) se reduce a la
      ecuación del calor. Este es, por tanto, un primer test que nos
      puede ayudar a garantizar que el código FreeFem++ que estamos
      programando es correcto.
    \end{quote}
  \end{flushright}
\item Representar la solución obtenida en los instantes de tiempo
  $t_i = k \cdot i$, para $i=0, 10, 20, 30, 40, 50$.
\end{enumerate}

\subsection*{Apartado 2.}
\begin{enumerate}
\item Demostrar la siguiente propiedad (conservación de la masa): si
  $u$ es solución de~(\ref{eq:flokker-planck}), entonces
$$
\int_\Omega u(\xx,t) \; dx = \int_\Omega u_0(\xx) \; dx, \quad \forall
t\in (0,T).
$$
  \begin{flushright}
    \begin{quote}\small \it
      Idea: aplicar la formulación
      variacional~(\ref{eq:formulacion-variacional-FP}) para la
      función test $v=1$.
    \end{quote}
  \end{flushright}
\item Realizar un programa FreeFem++, llamado \texttt{apartado2b.edp},
  en el que se aproxime la solución de~(\ref{eq:flokker-planck}) para
  los datos del apartado anterior salvo
  $$
  \V(\xx)=10\,u_0(x,y)  =  10\, e^{-10(x^2+y^2)}
  $$

\item Calcular (usando el operador \texttt{int2d} de FreeFem++) las
  integrales $\int_\Omega u(\xx,t_i) \; dx$, para todo instante
  $t_i=k\cdot i$, $i=0,\dots n_t$, imprimirlos en la salida estándar y
  comprobar que coinciden. Esto será un test de que el programa
  funciona correctamente (el conocimiento de la teoría matemática
  ayuda a programar correctamente!).
\item Representar, de nuevo, la solución obtenida en los instantes de
  tiempo $t_i = k \cdot i$, para $i=0, 10, 20, 30, 40, 50$. Comprobar
  que existen notables diferencias.
\end{enumerate}

\subsection*{Apartado 3.}
En este apartado, nos centraremos de nuevo en el caso en el que $\V$
es constante y por tanto~(\ref{eq:flokker-planck}) se reduce a la
ecuación del calor.

Llamaremos estado estacionario\footnote{
 Obsérvese que $u_\infty$ es la única solución constante de
$\grad u + u\grad\VV$ tal que
$\int_\Omega u_\infty \; dx = \int_\Omega u_0 \; dx$.}
de~(\ref{eq:flokker-planck}) a
$$
u_\infty = \frac 1{|\Omega|}\int_\Omega u_0(x)\; dx \ge 0,
$$
donde $|\Omega|=\int_\Omega 1 \;dx$ es la medida de
$\Omega$.  Definimos el
siguiente funcional:
$$
H_2[u] = \norm[2]{u-u_\infty}^2 =  \int_\Omega (u-u_\infty)^2 \; dx
$$
\begin{enumerate}
\item Denotamos $u(t)=u(\cdot,t):\Omega \to \Rset$, $t\in  [0,T]$.
  Demostrar que, para toda $u(t)$ solución $u(t)$
  de~(\ref{eq:flokker-planck}) con dato inicial $u_0$, se tiene la
  desigualdad
  \begin{equation}
    \label{eq:1}
    \frac{d}{dt} H_2 [u(t)] = -2 \int_\Omega |\grad u(t)|^2 dx \le 0
  \end{equation}
  \begin{flushright}
    \begin{quote}\small \it
      Idea: aplicar que, suponiendo suficiente regularidad,
      $\frac d{dt} H_2[u(t)] = \int_\Omega \frac d{dt} (u-u_\infty)^2$, a continuación usar la
      ecuación~(\ref{eq:flokker-planck})$_1$ e integrar por partes.
    \end{quote}
  \end{flushright}
  \begin{flushright}
    \begin{quote}\small \it
      Observación: Se dice que $H_2[u]$ (en realidad, $-H_2[u]$) es un
      funcional de entropía para~(\ref{eq:flokker-planck}). En
      general, un funcional de entropía es cualquier funcional que es
      no decreciente sobre las soluciones de una familia de EDP. Este
      tipo de funcionales son muy útiles para estudiar el
      comportamiento asintótico (cuando $t\to\infty$) de las
      soluciones. Ver~\cite{jungel_entropy_2016}.
    \end{quote}
  \end{flushright}
\item Crear un fichero \texttt{apartado3b.edp} (se puede tomar como
  base el programa \texttt{apartado2b.edp} del apartado anterior) para
  que se calculen, en cada iteración $i=0,...,n_t$, los valores de
  $H_2[u(t_i)]$. Utilizando la orden \texttt{ofstream} de FreeFem++,
  grabar estos datos en un fichero. Utilizando un programa externo,
  como \textit{Python}, \textit{Matlab} u \textit{Octave},
  representarlos gráficamente y confirmar el hecho de que
  $u\to u_\infty$ en norma $L_2$.

\end{enumerate}
\end{exercise}

\bibliographystyle{unsrt}
\bibliography{biblio}

\end{document}

%%% Local Variables:
%%% mode: latex
%%% TeX-master: t
%%% ispell-local-dictionary: "spanish"
%%% End:
